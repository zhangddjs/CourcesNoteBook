\subsection{模拟退火算法}

\paragraph{概述}~{}

以一定概率接受领域内次优解、随接受概率温度系数慢慢下降,有可能跳出局部最优。

\textbf{搜索算法:}盲目搜索还是启发式搜索?

按照预定的控制策略实行搜索,在搜索过程中获取的中间信息不用来改进控制策略,成为盲目搜索,反之称为启发式搜索。

\textbf{盲目搜索:}DFS、BFS、代价优先、向前、向后、双向。。。

\textbf{启发式搜索:}爬山法、\textbf{模拟退火法(SA)}、\textbf{遗传算法(GA)}、\textbf{粒子群算法}、禁忌搜索法(TS)。。。

\textbf{爬山法:}爬山法在贪心法的基础上新解数量从一个变成多个,能够\textbf{快速收敛于局部最优解}。但是遇到\textbf{平台}则无以是从,对初始解\textbf{敏感},容易\textbf{陷入}局部最优。

\textbf{模拟退火法:}

物理退火过程:粒子状态 \rightarrow 能量最低态 \rightarrow 加温过程 \rightarrow 等温过程 \rightarrow 冷却(退火)过程 \rightarrow 能量

模拟退火过程:解 \rightarrow 最优解 \rightarrow 设定初温(加温) \rightarrow Metropolis采样过程(等温) \rightarrow 控制参数的下降(冷却) \rightarrow 目标函数

优点是原理简单,计算速度快。

\paragraph{应用场景}~{}

求组合优化问题、求函数最优解问题。例如TSP问题、调度问题。

这类问题在找全局最优解时需要彻底搜索整个可行域,同时对于某些情况有大量可行解。

\paragraph{算法}~{}

模拟退火算法如算法\ref{algo:simulatedannealing}所示。

\begin{algorithm}[htb]
  \small
  \SetAlgoLined
  \KwData{数据集}
  \KwResult{最优点}

  随机产生一个初始解$x_0$,令$x_{best}=x_0$,并计算目标函数值$E(x_0)$\;
  设置初始温度$T(0)=T_0$,迭代次数$i=1$\;
  \While{$T(i)>T_{min}$}{
    \For{$j=1\sim{}k$}{
        对当前最优解$x_{best}$按照某一邻域函数,产生一个新解$x_{new}$。计算新的目标函数值$E(x_{new})$,并计算目标函数值的增量$\Delta{E}=E(x_{new})-E(x_{best})$\;
        \lIf{$\Delta{E}<0$}{$x_{best}=x_{new}$}
        \If{$\Delta{E}>0$}{
            $p=exp(-\Delta{E}/T(i))$\;
            \leIf{$c=random[0,1]<p$}{$x_{best}=x_{new}$}{$x_{best}=x_{best}$}
        }
    }
  }
  输出当前最优点,计算结束\;

  \caption{模拟退火算法}
  \label{algo:simulatedannealing}
\end{algorithm}

\paragraph{要素}~{}

\begin{enumerate}
    \item 状态空间与状态产生函数(邻域函数)
    \item 状态转移概率(接受概率)$p$:$x_{old}$向$x_{new}$转移的概率。采用\textbf{Metropolis准则}
    $$p=\left\{\begin{matrix}
        1 & if & E(x_{new})<E(x_{old}) \\ 
        exp(-\frac{E(x_{new})-E(x_{old})}{T}) & if & E(x_{new})\geq{}E(x_{old})
      \end{matrix}\right.$$
    \item 冷却进度表T(t),经典方式为$T(t)=\frac{T_0}{\lg(1+t)}$,快速方式为$T(t)=\frac{T_0}{1+t}$。这两种方式都可以收敛于全局最小点。
    \item 初始温度$T_0$,初温越大,获得高质量解概率越大,计算量越大。确定方式主要有三种。
    \item 内循环终止准则(Metropolis抽样稳定准则),决定各温度下产生候选解的数目,常用的有三种。
    \item 外循环终止准则,常用的有四种
\end{enumerate}

\paragraph{优化}~{}

\begin{enumerate}
    \item \textbf{增加升温或重升温过程。}(避免在局部最小解停滞不前)
    \item \textbf{增加记忆功能。}(避免由于执行概率接受环节而遗失当前遇到的最优解,记录Best So Far状态)
    \item \textbf{增加补充搜索过程。}(退火结束后以当前最优解为初始状态再搜索一次)
    \item 对每一当前状态采用多次搜索策略,以概率接受区域内的最优状态。
    \item 结合其它搜索算法,如遗传、混沌等。
\end{enumerate}
