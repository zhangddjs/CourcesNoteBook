\subsection{计算机仿真}

\paragraph{概述}~{}

计算机仿真是用计算机对一个系统的结构和行为进行动态演示,以\textbf{评价或预测}一个系统的行为效果,为决策提供信息的一种方法。分为离散和连续的系统仿真。

\textbf{使用理由:}

\begin{enumerate}
    \item 在实际系统没有建立之前,仿真结果可以方便研究系统行为和结果
    \item 在有些真实系统上做实验会影响系统正常运行。
    \item 人是系统一部分时,他的行为往往会影响试验结果。
    \item 在实际系统上多次试验时,难以保证每次操作条件一致,因此难以判断结果好坏。
    \item 有些试验时间太长,费用太大,或有危险。
    \item 有些系统一旦建立就无法复原。
\end{enumerate}

\textbf{要素:}

\begin{itemize}
    \item \textbf{系统:}一些具有特定的功能相互之间以一定规律联系着的物体所组成的整体。
    \item \textbf{实体:}系统的对象、系统的组成元素都是系统的实体。
    \item \textbf{属性:}反应了实体的某些性质,可以是文字型、数字型或逻辑型。
    \item \textbf{状态:}系统的状态指某一时刻实体以及其属性的集合。
    \item \textbf{事件:}活动是指一段过程情况,事件指时间点的情况。
    \item \textbf{仿真时钟:}研究系统状态随时间变化的规律,需要仿真一个时间变量,对连续情况则提供均匀时间点的解,对离散情况记录事件发生的解。
    \item \textbf{事件表:}某些离散事件的仿真过程中,可以采用事件表进行调度。事件表一般是一个有序的记录列,记录时间-事件。
\end{itemize}

\textbf{步骤:}

系统分析 \rightarrow 模型构造(时间步长法、事件表法等) \rightarrow 模型的运行与改进 \rightarrow 设计格式输出仿真结果

\paragraph{应用场景}~{}

需要进行模拟的系统,系统状态随时间变化,随机现象模拟,求最优解。

交通灯时间、航空管理、道路修建、公交车调度、减速墩、三峡的安全、生态、用药量、新药的开发(成分的比例)、医疗保险、国债的发行、邮票面值、动画设计、家居装修、炼钢的温度估计、炼油、发电厂的操作模拟训练、鼠疫的检测和预报、CT。

\paragraph{案例}~{}

1. 混凝土搅拌中心的位置:需要运送混凝土到n个工地,每吨每公里C元,如何设置搅拌中心所在的工地位置,使得费用最少?

这个问题可以用数学的方法求出,同时也能用计算机仿真程序循环计算模拟搅拌机放在每个位置的费用,比较得出最佳位置。

2. 求重叠面积

重叠面积可以用蒙特卡罗方法,同时也可以用计算机仿真来求解,将平面分成一个个小方格,相加即可。

3. 求不断注水的水池含盐量问题

根据公式用时间步长法或事件表法模拟含盐量走势,从而得到结果。
