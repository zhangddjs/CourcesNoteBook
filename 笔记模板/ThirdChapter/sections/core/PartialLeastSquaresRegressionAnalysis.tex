\subsection{偏最小二乘回归分析}

\paragraph{概述}~{}

研究两组或多重相关变量间的相互依赖关系,用自变量(预测变量)去\textbf{预测}因变量(响应变量)。

常用方法:主成分回归分析(PCR)、最小二乘准则下的经典多元线性回归分析(MLR)、偏最小二乘(PLS)。

\paragraph{应用场景}~{}

偏最小二乘回归提供一种多对多线性回归建模方法,\textbf{两组变量}个数很多,且都存在\textbf{多重}相关性,\textbf{样本量少}。

例如:身体素质(身高、体重等) \rightarrow 体能训练结果

\paragraph{算法}~{}

经典算法如算法\ref{algo:plsalg}所示。

\begin{algorithm}[htb]
    \small
    \SetAlgoLined
    \KwData{自变量集$X$; 因变量集$Y$}
    \KwResult{达到满意精度的回归方程}
  
    提取$X$,$Y$的第一对成分$t_1$,$u_1$,并使$t_1$与$u_1$变异信息和关联程度最大\;
    \Repeat{回归方程精度没有达标}{
        建立$X$,$Y$对t1的回归模型\;
        优化残差阵\;
        得到r个成分,取前l个成分,解得$Y$的偏最小二乘回归方程\;
        交叉有效性检验,要求抽取h个成分后,$PRESS(h)/SS(h - 1)$越小越好\;
    }
  
    \caption{偏最小二乘回归分析算法}
    \label{algo:plsalg}
\end{algorithm}

有一种更简单的算法,直接在残差阵提取成分$t_1 \sim t_r$,无需提取$u_h$。
